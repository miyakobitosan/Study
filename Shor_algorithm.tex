\documentclass{jsarticle}
\usepackage{amsmath,amssymb}
\usepackage{amsthm}
%\usepackage{pdfpages}
\usepackage[dvipdfmx]{graphicx}
\usepackage{color}
%\usepackage{mediabb}
\theoremstyle{definition}
\newtheorem{theorem}{定理}
\newtheorem{definition}[theorem]{定義}
\newtheorem{lemma}[theorem]{補題}
\renewcommand\proofname{\bf 証明}
\begin{document}
\section{Shor's algorithm}
\subsection{アルゴリズムの目標}
$M \in \mathbb{Z}_{>0}$に対して, 次の$2$つの条件を満たす函数$f:\mathbb{N}\rightarrow \{0, M-1 \}$を考える.\\
【条件】\\
$1.$ 周期 $^{\exists !}r$	$($i.e$)$ for $^{\forall} x \in \mathbb{N}, $ $f(x+r)=f(x)$\\
$2.$ $f(0), \cdots , f(r-1)$ are all distinct. 

\subsection{アルゴリズムの戦略}
まず, 記号の準備をする. for $a \in \mathbb{N}$, let $f_a:\mathbb{N}\rightarrow \{ 0, \cdots , M-1 \} , x\mapsto a^x \mod M.$ \\
$1.$ $M ( n$  bit $)$ の整数,  $a \in \{1, M-1 \}$ を gcd $(a, M)=1$を満たす整数とする.\\
$2.$ 等重の並列状態にする. $\frac{1}{\sqrt{Q} } \displaystyle\sum_{x}^{Q-1}  |x  \rangle$ \\
$3.$ Quantum part and measure. \\
$4.$ 周期$r$を$3.$で得た$x$から推定する. $x$が "good number " なら, $r$を推測できる. そうでなければ,再度$3$に戻る.\\

 

\end{document}