\documentclass{jsarticle}
\usepackage{amsmath,amssymb}
\usepackage{amsthm}
%\usepackage{pdfpages}
%\usepackage{mediabb}
\usepackage[dvipdfmx]{graphicx}
\usepackage{color}
%\usepackage{mediabb}
\theoremstyle{definition}
\newtheorem{theorem}{定理}
\newtheorem{definition}[theorem]{定義}
\newtheorem{lemma}[theorem]{補題}
\renewcommand\proofname{\bf 証明}
\newtheorem{problem}[theorem]{問題}
\begin{document}
Review of the Simon's Algorithm
\section{記号}
\begin{definition}
$N=2^n$として
\begin{align}\label{Hadamar}
H_N=H_{N/2}\otimes H_2
=\frac{1}{\sqrt{2} } \left( \begin{array}{cc}
H_{N/2} & H_{N/2} \\
H_{N/2} & -H_{N/2} 
\end{array}\right)
\end{align}
と定める. ここで, $H_1=1$とする.
\end{definition}
アダマール変換の一般系$H_N$について次の補題が成立する.
\begin{lemma}
For row  $r=(r_1,r_0) \in \{0,1\}^{N-1} \times \{0,1\}$ and column $c=(c_1,c_0) \in \{0,1 \}^{N-1} \times \{0,1\}$, 
\begin{align}
H_{N}[r,c]=(-1)^{r \bullet c }
\end{align}
\end{lemma}
\begin{proof}
一般に $n \times n$行列$A$に対して, $(i,j)$成分は $A\mathbf{e}_j $の第$i$成分として求まる. よって, $(r,c)$成分を求めるには, $H_{N} \mathbf{e}_{c_1} \otimes \mathbf{e}_{c_0} $の$r$成分を求めると良い.
\begin{align}
\begin{array}{lll}
H_{N} \mathbf{e}_{c_1} \otimes \mathbf{e}_{c_0}&=&H_{N/2} \mathbf{e}_{c_1}\otimes H_2  \mathbf{e}_{c_0} 
\end{array}
\end{align}
であるので, $ r=(r_1,r_0) \in \{0,1\}^{N-1} \times \{0,1\} $成分は, 帰納法の仮定から, $(-1)^{r_1 \bullet c_1} \cdot (-1)^{r_0 \bullet c_0}=(-1)^{r\bullet c}$とわかる.
\end{proof}

\section{Simon Algorithm}
Let $f:\{0,1\}^n \rightarrow \{0,1\}^n $. We  assume two condition;\\
$(1)$ there is a $s$  (s.t)  for  $^{\forall}y, ^{\forall}z $,    $f(y)=f(z) \Leftrightarrow  y=z \oplus s$\\
$(2)$ $f$ is a one to one or two to one.
\ \\
また, $x\in \{0,1 \}^n$に対して,  $x^* : \{0,1\}^n \rightarrow \{0,1\}$ を $v \mapsto  x \bullet v $と定義する. また, 函数$h: \{0,1\}^n \rightarrow \{0,1\} $に対して, $V(h):=\{x \in \{0,1\}^n | h(x)=0 \}$と定義をする.
\  \\
【戦略】\\
$1$.  $V(0)$\\
$2$-$1$.  $\frac{1}{\sqrt{N}} \displaystyle\sum_{x\in \{0,1\}^n } \mathbf{e}_x \otimes \mathbf{e}_{f(x)}$ \\  
$2$-$2$. 第一レジスタに対して, Hadamard 変換をし, 測定をし, 第一レジスタ $x_i$ とする. \\
$2$-$3$. $V(x^*_1, \cdots , x^*_i)=s$となるまで $2$-$1$, $2$-$2$を繰り返す. \\
\section{解析} 
\begin{lemma}
Suppose $f$ is periodic with nonzero $s$. Then the measured $x$'s are random Boolean strings in $\{0,1\}^n$ such that $x\bullet s=0$.
\end{lemma}
\begin{proof}
Step $2$-$2$で,  Hadamard 変換を施すと, 
\begin{equation}
\frac{1}{N}\displaystyle\sum_{x\in \{0, 1 \}^n }\displaystyle\sum_{t\in \{ 0,1 \}^n } (-1)^{x\bullet t} \mathbf{e}_t \otimes \mathbf{e}_{f(x)}
\end{equation}
であり, $f$ periodic と仮定しているので, $^{\exists}x_1, x_2=x_1\oplus s $ $($s.t$)$  $f(x_1)=f(x_2)$. このとき, $e_t \otimes e_{f(x_1)}$の係数は, 
\begin{equation}
\frac{1}{N}(-1)^{t\bullet x_1}(1+ (-1)^{x \bullet s}).
\end{equation}
なので, 第二レジスタが Range$(f)$の時のみ, 確率$\frac{4}{N^2}$で第一レジスタ$t$に対して, $t\bullet s=0$
\end{proof}
\section{研究課題}
最短でどのくらいか見積もってみる.
\end{document}