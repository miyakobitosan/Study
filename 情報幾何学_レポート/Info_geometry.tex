\documentclass{jsarticle}
\usepackage{amsmath,amssymb}
\usepackage{amsthm}
%\usepackage{pdfpages}
%\usepackage{mediabb}
\usepackage[dvipdfmx]{graphicx}
\usepackage{color}
%\usepackage{mediabb}
\theoremstyle{definition}
\newtheorem{theorem}{定理}
\newtheorem{definition}[theorem]{定義}
\newtheorem{lemma}[theorem]{補題}
\renewcommand\proofname{\bf 証明}
\newtheorem{problem}[theorem]{問題}
\begin{document}
\section{準備:幾何学}
\begin{definition}
Let $X$  be a set. A topology on $X$ is a collection of subsets of $X$, such that:\\
$(1)$ the empty set $\phi$ and the set $X$ are open;\\
$(2)$ the union of an arbitrary collection of open sets is open;\\
$(3)$ the intersection of a finite number of open sets is open.
\end{definition}
\begin{problem}
Xの部分集合族$\mathcal{O}$を用いて、上記の位相空間の定義を書き直せ.
\end{problem}
\begin{definition}
位相空間$M$が次の条件$(1), (2)$を満足する時, $M$を$m$ dimensional topological manifoldという.\\
$(1)$ $M$ is Hausdorff space.\\
$(2)$ For $^{\forall} p\in M$  $m$ 次元座標近傍$(U, \varphi)$ が存在する.
\end{definition}
\begin{definition}
Given any $C^k$ manifold $M$, of dimension $n$, with $k \geq 1$, for any $p\in M$ , a tangent vector to $M$ at $p$ is any equivalence class of $C^1$ curves through $p$ on $M$, modulo the equivalence relation defined in below.
\begin{equation}
\gamma_1:(-\epsilon _1, \epsilon_1 ) \rightarrow M  \gamma_2:(-\epsilon _2, \epsilon_2 ) \rightarrow M \text{ are equivalent } \Leftrightarrow ^\exists (U, \varphi ) \text{  s.t } \frac{d(\varphi \circ \gamma_1)}{dt} (0)=\frac{d(\varphi \circ \gamma_2)}{dt} (0)
\end{equation}
\end{definition}
次に逆写像定理が重要となる.
\begin{theorem}
$M,N$を$C^r$級多様体, $f:M\rightarrow N$ を$C^r$級写像とする.
$(df)_p:T_p(M)\overset{\sim}{\longrightarrow }T_{f(p)}N $なら, $p \in ^\exists U, f(p) \in ^\exists V $ s.t $f|U:U\overset{\sim}{\longrightarrow}V.$
\end{theorem}

\end{document}